%-----------------------------------------------------------------
% PSE 02 - Team Donnervogu
%
% Filename: Vortrag.tex
% Date: 11.04.2011
% Author: Felix Langenegger <felix.langenegger@gmail.com>
% Edited by Jonas Ruef <jonas.ruef@gmail.com>
%-----------------------------------------------------------------
\documentclass{beamer}
% Latex Packages
\usepackage[utf8x]{inputenc}
\usetheme{Singapore}
\usepackage[ngerman]{babel}
\usepackage{graphicx}
% Standard Beamer Latex Settings
\setbeamercovered{dynamic}
% Frame Numbers on
\setbeamertemplate{footline}[frame number]
% Set Beamer Colors
\setbeamercolor{normal text}{fg=white,bg=black}
\setbeamercolor{structure}{fg=white}
\setbeamercolor{section in head/foot}{fg=white}
\setbeamercolor{section in head/foot shaded}{fg=gray}
% Set Fontsize in Head and Foot
\setbeamerfont{section in head/foot}{size={\fontsize{6}{01}}}

%-----------------------------------------------------------------
%Edit here
%-----------------------------------------------------------------
\title{Architektur \& Technologie}
\author{PSE Team 2\\Jonas Ruef - Chief Deliverable Officer}
\date{11. April  2011}

\begin{document}

\maketitle
\frame{
	\frametitle{Inhaltsverzeichnis}
	\tableofcontents
	%[pausesections]
}
\begin{frame}{Ruby 1.9.2}
\section{Technologie}
\subsection{Ruby 1.9.2}
\bigbreak
\noindent\includegraphics[width=9cm, keepaspectratio]{Ruby}
\end{frame}
%---------------------------------
\begin{frame}{Ruby on Rails 3.0.5}
\subsection{Ruby on Rails 3.0.5}
\begin{itemize}
 \item Web Application Framework
 \item Vollständig in Ruby geschrieben.
 \item Webserver: WEBrick, Apache, Lighttpd, FastCGI
 \item \textit{Scaffolding}
\end{itemize}
\end{frame}
%---------------------------------
\begin{frame}{Architektur}
\section{Architektur}
\begin{itemize}
 \item Client (PSE01) - Server (PSE02)
 \item MVC Pattern in Rails
\end{itemize}
\end{frame}
%---------------------------------

\begin{frame}{Model}
\subsection{Model}
\begin{itemize}
 \item Datenbank (SQLite, MySQL, PostgreSQL, etc)
 \item Separate Datenbank für: Tests, Entwicklung und den produktiven Einsatz
 \item Zugriff auf Datenbank mit Hilfe von ActiveRecord
\end{itemize}
\end{frame}
%---------------------------------
\begin{frame}{View}
\subsection{View}
\begin{itemize}
\item HTML \& ERB oder Haml?
\bigbreak
\noindent\includegraphics[height=2.39cm, keepaspectratio]{Haml}
\bigbreak
\bigbreak
\item Konsequente Verwendung von Haml
\end{itemize}
\end{frame}
%---------------------------------
\begin{frame}{Controller}
\subsection{Controller}
\begin{itemize}
\item Controller Funktionen in Rails
 \item  Standard Rails-Request: servername.net/controller/action
 \item  Veränderbar mittels selbstdefinierter Routen
\end{itemize}


\end{frame}
%----------------------------------
\begin{frame}{Fragen}
%\section{Fragen}
 \begin{center}
\fontsize{60}{80}\selectfont ?
 \end{center}
\end{frame}
\end{document}
%----------------------------------
