%-----------------------------------------------------------------
% PSE 01 - Team Donnervogu
%
% Filename: testkonzept.tex
% Date: 25.03.2011
% Author: Aaron Karper <akarper@students.unibe.ch>
%-----------------------------------------------------------------
\documentclass{beamer}
% Latex Packages
\usepackage[utf8x]{inputenc}
\usetheme{Singapore}
\usepackage[ngerman]{babel}
% Standard Beamer Latex Settings
\setbeamercovered{dynamic}
% Frame Numbers on
\setbeamertemplate{footline}[frame number]
% Set Beamer Colors
\setbeamercolor{normal text}{fg=white,bg=black}
\setbeamercolor{structure}{fg=white}
\setbeamercolor{section in head/foot}{fg=white}
\setbeamercolor{section in head/foot shaded}{fg=gray}
% Set Fontsize in Head and Foot
\setbeamerfont{section in head/foot}{size={\fontsize{6}{01}}}
%-----------------------------------------------------------------
%Edit here
%-----------------------------------------------------------------
\title{Testkonzept}
\author{PSE Team 2 \\ Aaron Karper}
\date{28. März 2011}

\begin{document}

\maketitle
\begin{frame}{Testkonzept}
	\tableofcontents
\end{frame}
\section{Erwägungen}
\begin{frame}{Erwägungen}
	\begin{itemize}
	  \item Rolle
	  \item Umfang
	  \item Fokus
	  \item Technik
	\end{itemize}
\end{frame}
\subsection{Analyse}
\begin{frame}{Analyse}
  \begin{itemize}[<+->]
    \item Serverapplication
    \item GUI tiefe Priorität
    \item Komponenten nicht sinnvoll trennbar
    \item Fixe Anzahl von Clients
  \end{itemize}
\end{frame}
\subsection{Testarten}
\begin{frame}{Testarten}
  \begin{itemize}[<alert@+>]
    \item Unittests
    \item Datenbanktest
    \item GUI- \& Usability-Test
    \item Installationstest
    \item Stress-Test
    \item Integrationtests
  \end{itemize}
\end{frame}
\section{Konzept}
\subsection{Welche Tests?}
\begin{frame}{Konzept: Welche Tests?}
  \begin{itemize}[<+->]
    \item Fokus auf Integrationtests
    \item Unittests im Modell soweit möglich
    \item Datenbank nur indirekt
  \end{itemize}
\end{frame}
\subsection{Wie getestet?}
\begin{frame}{Konzept: Wie wird getestet?}
    Nutzen von Rails' Testing-Framework
  \begin{itemize}
    \item Unit-, Integration-, Performance, \ldots
    \item Gut integriert
  \end{itemize}
\end{frame}
\subsection{Wie geschrieben?}
\begin{frame}{Konzept: Wie geschrieben?}
  \begin{itemize}
    \item Teampartner schreibt Tests
    \item Gleichzeitig oder kurz nach Implementierung
    \item Kein TDD
  \end{itemize}
\end{frame}
\section{Fragen}
\begin{frame}{Fragen?}
\end{frame}
\end{document}
