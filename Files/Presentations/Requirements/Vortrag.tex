%-----------------------------------------------------------------
% PSE 02 - Team Donnervogu
%
% Filename: Vortrag_1.tex
% Date: 01.03.2011
% Author: Felix Langenegger <felix.langenegger@gmail.com>
% Edited by: Jonas Ruef
%-----------------------------------------------------------------
\documentclass{beamer}
% Latex Packages
\usepackage[utf8x]{inputenc}
\usetheme{Singapore}
\usepackage[ngerman]{babel}
% Standard Beamer Latex Settings
\setbeamercovered{dynamic}
% Frame Numbers on
\setbeamertemplate{footline}[frame number]
% Set Beamer Colors
\setbeamercolor{normal text}{fg=white,bg=black}
\setbeamercolor{structure}{fg=white}
\setbeamercolor{section in head/foot}{fg=white}
\setbeamercolor{section in head/foot shaded}{fg=gray}
% Set Fontsize in Head and Foot
\setbeamerfont{section in head/foot}{size={\fontsize{6}{01}}}
%-----------------------------------------------------------------
%Edit here
%-----------------------------------------------------------------
\title{Requirements}
\author{PSE Team 2\\Jonas Ruef
\\Felix Langenegger}
\date{07. März 2011}

\begin{document}

\maketitle
\frame{
	\frametitle{Inhaltsverzeichnis}
	\tableofcontents
	[pausesections]
}
\begin{frame}{Kunde}
\section{Einführung}
\subsection{Kunde}
\begin{itemize}
 \item Michael Waelti
\end{itemize}
\end{frame}
%---------------------------------
\begin{frame}{Team PSE 2}
\subsection{Team PSE 2}
\begin{itemize}
 \item \textbf{Aaron Karper} - Quality Evangelist
 \item \textbf{Dominique Rahm} - Key Account Manager
 \item \textbf{Felix Langenegger} - Master Tracker
 \item \textbf{Jonas Ruef} - Chief Deliverable Officer
 \item \textbf{Sascha Schwärzler} - Head of Communication
\end{itemize}
\end{frame}
%---------------------------------
\begin{frame}{Coach}
\subsection{Coach}
\begin{itemize}
 \item Max Leske
\end{itemize}
\end{frame}
%---------------------------------
\begin{frame}{Projekt}
\section{Projekt}
\begin{itemize}
 \item Thunderbird Plugin
 \item Automatisches managen von Signaturen, Formatiereungen, etc.
 \item zentrales Management
\item Team 1 und Team 2 haben dasselbe Projekt
\end{itemize}

\end{frame}
%---------------------------------
\begin{frame}{Problem?!}
\section{Aufteilung}
Wie teilt man ein Projekt auf zwei Teams auf?
\end{frame}
%---------------------------------
\begin{frame}{Szenario 1}
\subsection{Szenario 1}
Jedes Team setzt das Projekt individuell um. Keine Absprache unter den beiden
Teams.\\ \bigbreak
\textbf{Vorteile}:
\begin{itemize}
 \item Kunde erhält \textbf{\underline{zwei}} Produkte und kann dann
entscheiden
 \item kleiner Kommunikationsaufwand
 \item ev. Wettbewerb zwischen den beiden Teams
\end{itemize} \bigbreak
\textbf{Nachteile}: 
\begin{itemize}
 \item Doppelspurigkeit
 \item Kunde muss an zwei Meetings
 \item ev. wird kein Projekt richtig fertig
\end{itemize}
\end{frame}
%---------------------------------
\begin{frame}{Szenario 2}
\subsection{Szenario 2}
Beide Teams arbeiten am selben Projekt.\\ \bigbreak
\textbf{Vorteile}:
\begin{itemize}
 \item Kunde bekommt mehr Stories umgesetzt
 \item Nur ein Kundenmeeting
\end{itemize} \bigbreak
\textbf{Nachteil}:
\begin{itemize}
 \item Grosser Kommunikationsaufwand
\end{itemize}
\end{frame}
%---------------------------------
\begin{frame}{Definitive Aufteilung}
\subsection{Definitive Aufteilung}
\begin{itemize}
 \item Der Kunde wünschte sich Szenario 2
 \item Ein Team entwickelt das Thunderbird Plugin
 \item Das andere Team die Serverapplikation und das Webinterface für den Admin
\end{itemize}
Welches Team welchen Part übernimmt ist noch Teil der wöchentlichen Abklärung.
\end{frame}
%---------------------------------
%---------------------------------
\begin{frame}{Kunde}
\section{Kunde}
Michael Waelti\\
NILE Clothing AG, NILE Trading, Bijou les Boutiques
\bigbreak
\begin{itemize}
\item Fadendaten interne IT der drei Firmen
\item Externe IT: Aarboard
 \item Noch Master Student an der Universität Bern
 \item Studierte im Nebenfach Informatik
\end{itemize}
\end{frame}
%---------------------------------
\begin{frame}{Interne IT- Infrastruktur Kunde}
\subsection{Interne IT- Infrastruktur Kunde}
\begin{itemize}
\item 2 Logistikserver
\item 1 Testserver
\item Ca. 70 Clients
\bigbreak
\end{itemize}
Klein aber schnell expandierend!\\
Lösung muss einfach erweiterbar und skalierbar sein.
\end{frame}
%---------------------------------
\begin{frame}{Kundenmeeting}
\subsection{Kundenmeeting}
Technisch versierter Kunde: Fluch oder Segen?
\bigbreak
Problematik mit 2 Teams: Mit dem Kunden 12 Leute an Meetings!!!
\end{frame}
%---------------------------------
\begin{frame}{Requirements dieser Woche}
\subsection{Requirements dieser Woche}
Noch kein Planning Game durchgeführt.
\bigbreak
Recherche von:
\begin{itemize}
\item Technologien von Thunderbird-Plugin
\item Technologien auf Server
\item Schnittstelle zwischen Server und Plugin
\end{itemize}
\end{frame}
%---------------------------------
\begin{frame}{Fragen}
\section{Fragen}
 \begin{center}
\fontsize{60}{80}\selectfont ?
 \end{center}
\end{frame}
\end{document}
